\begin{FlushLeft}
\section*{Content and Presentation}

\subsubsection*{Quality of Text Content}

The text of the MIT website is well-written and generally accessible. Understandability is
fine, however it is sometimes difficult to sift through all of the information provided to
find what is essential to the key questions of cost and degree requirements.

The Stanford site has well-written text and the site makes good use of whitespace
in text-heavy sections. There is at least one spelling error on the computer science
department page, but it is not a common occurence on the site.
\end{FlushLeft}

\subsubsection*{Font Size and Page Color}



% note in the margin
\linespread{1.0}
\begin{wrapfigure}[6]{r}{.2\textwidth}
\colorbox{white}{\begin{minipage}{0.25\textwidth}\begin{scriptsize}\begin{center}
			 	   The MIT site was quite challenging for me to read, and although I recognize that my visual impairment makes me less
				   than objective on this, it certainly was a challenge for me to read the content of the site.\par

				   \textbf{Samuel Barton}\end{center}\end{scriptsize}
\end{minipage}}
\end{wrapfigure}

\linespread{1.3}
		The fonts chosen by the MIT website are on the whole far too small and difficult to read.
		The background coloring is usually fine, on some of the pages the contrast is done in an
		odd way that is less than useful, but these are far down the hierarchy on pages devoted to
		small departments or individuals and thus do not reflect the site as a whole.

\setlength{\parindent}{0pt}
Stanford uses a sans serif and thin font throughout their site. It can be hard to read,
especially on ultra-high resolution screens. However, the color scheme for text
content greatly improves readability. Backgrounds are white or off-white, and text
is either dark or cardinal. The darker foreground colors make text pop while
reinforcing the brand of the university---as cardinal and white are Stanford's colors.

\subsubsection*{Error Messages and Progress Reports}

\begin{FlushLeft}

Very few to no error messages are presented on the MIT and Stanford sites.
Progress reports are entirely unused throughout the sites as the sites are dedicated to
information proliforation. Beyond that, the places where a progress report would be
useful are inaccessible to the general populace as they would exist within the student or
faculty portals.

\subsubsection*{Images --- Usefulness}

Images are used sparingly throughout the MIT site, and mostly consists of headshots of people who were
being highlighted for various reasons. Thus the images support the text they are used
with, but they are never used to help deliver the useful information which the site
is attempting to provide.

The Stanford site uses images often in three ways: \textit{(1)} to provide an image
of a person that is being talked about in a section, \textit{(2)} to provide high-quality
images of the Stanford campus or smiling actors, especially on the higher levels of
the site and in the parent gateway, and \textit{(3)} to provide an image of a building,
office, device, or lab being discussed, especially in the School of Engineering
pages. These images are always useful and support the text around them.

\subsubsection*{Images --- Accessibility}

All images that would reasonably have captions do on MIT's website, background images do not but
adding captions to them would make the site less usable so that is a perfectly reasonable
decision. We only found one image that had a tooltip, and
about half of the images had \texttt{alt} text. When the \texttt{alt} text was present it was generally quite
useful, but it was not present several times when it would have been incredibly helpful to
someone who was blind, had a slow internet connection, or was dealing with tight filtering.

Images on the front pages of Stanford's site have both tooltips and sufficient
\texttt{alt} attributes. However, images on individual school pages and especially
pages for smaller organizations within the university lack both tooltips and
\texttt{alt} attributes. For example, at the time of this writing, every image
on the Stanford computer science department home page (\url{https://www-cs.stanford.edu/})
is missing both tooltips and \texttt{alt} attributes.

Most if not all images on the Stanford site lack visible text captions.

\subsubsection*{Presentation of Graphics}

The MIT site is incredibly minimalistic. It borders on dull while maintaining just enough flair
to seem interesting. This is obviously only considering the home portal. On average the
subordinate pages are much more interesting graphically with interactive menus,
three-dimensional backgrounds, and so on. There are some pages, the philosophy department's
page in particular, which are entirely drab as they are nothing but plaintext HTML with no
styling.

On the whole the site is somewhat dull, with interesting parts where it is likely many
people would be interacting, and incredibly plain parts where they are less likely to
frequent. This was highly dissappointing as MIT is considered one of the premiere
institutions of higher learning for technology and the sciences, but they cannot build an
attractive website.

In contrast, Stanford's site is attractive throughout. Stanford's site is minimalistic
like MIT's, but leverages the university's brand and look and feel to create a unified,
polished presentation. Unlike MIT, this polish persists over most of the site's
hierarchy; it's not until you reach the smaller leaf pages of the hierarchy do you
lose the signature style and presentation.

There's an interesting note to make about both the MIT and Stanford sites that wouldn't
have been relevant just a few years ago. Both of the sites use custom images as buttons
and icons. We observed that these images appear fuzzy and low fidelity on a MacBook
Pro Retina display. These ultra-high resolution displays are often sensitive to
bitmap images, and an image that is acceptable on even an HD monitor might appear
low quality on a Retina display. It could potentially be expensive to commission
higher quality or vector updates for all images on these sites, but as even lower-end
devices start to get higher quality displays, these sites will lose more visual
quality for more users.

\subsubsection*{Color}

Color is very minimal on the MIT site; most of the pages are white with black text. There
are colors on some pages, and when they are present they are tastefully presented. But
even with all the variance among the departments on the site, there are next to none
which employ color beyond the very minimum needed to differentiate menus from the main
body of their pages.

On the Stanford site, color is used in two ways: to emphasize important text,
and to make widgets and navigation tools visually pop off the background.
For the most part, Stanford's school colors---cardinal and white---are used,
along with dark grey accents. This consistency unified the visual presentation
of the site.

\subsubsection*{Audio and Video}

Sound is not present on the MIT site, and beyond a few video clips from interviews, or the
showcasing of experiments on some research pages there is no video either. The videos
present are perfectly acceptable, and more often than not interesting, but there are very
few of them.

Most videos on the Stanford site are embedded YouTube videos, and always have ample
padding around them so that they do not clutter the page. Videos are kept to a
minimum and if a page contains an embedded video, its presence is usually the
primary purpose of the page.

\subsubsection*{Privacy and Monitoring}

The MIT site has no privacy policy per se, at least on the portions of the site accessible to the
general public. Data collection is either nonexistant, or so
minimal it escapes detection even with repeated visits to the site.

Likewise, the Stanford website has no available privacy policy, and makes no mention
of whether the site records data about visits.
\end{FlushLeft}
