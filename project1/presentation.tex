\section*{Content and Presentation}

\subsubsection*{Quality of Text Content}

The text of the MIT website was well-written and generally accessible. Understandability was
fine, however it was sometimes difficult to sift through all of the information provided to
find what was essential to the key questions of cost and degree requirements.

The Stanford site has well-written text and the site makes good use of whitespace
in text-heavy sections. There is at least one spelling error on the computer science
department page, but it is not a common occurence on the site.

\subsubsection*{Font Size and Page Color}

The fonts chosen by the MIT website were on the whole far too small, they were quite
challenging for me to read, and although I recognize that my visual impairment makes me less
than objective on this, it certainly was a challenge for me to read the content of the site.

The background coloring was usually fine, on some of the pages the contrast was done in an
odd way that was less than useful, but these were far down the hierarchy on pages devoted to
small departments or individuals and thus do not reflect the site as a whole.

Stanford uses a sans serif and thin font throughout their site. It can be hard to read,
especially on ultra-high resolution screens. However, the color scheme for text
content greatly improves readability. Backgrounds are white or off-white, and text
is either dark or cardinal. The darker foreground colors make text pop while
reinforcing the brand of the university---as cardinal and white are Stanford's colors.

\subsubsection*{Error Messages and Progress Reports}

Very few to no error messages were presented on the MIT and Stanford sites.
Progress reports were entirely unused throughout the sites as the sites are dedicated to
information proliforation. Beyond that, the places where a progress report would have been
useful are inaccessible to the general populace as they would exist within the student or
faculty portals.

\subsubsection*{Images --- Usefulness}

Images were used somewhat sparingly throughout the MIT site, and mostly consisted of headshots of people who were
being highlighted for various reasons. Thus the images supported the text they were used
with, but they were never used to help deliver the useful information which the site
was attempting to provide.

The Stanford site uses images often in three ways: \textit{(1)} to provide an image
of a person that is being talked about in a section, \textit{(2)} to provide high-quality
images of the Stanford campus or smiling actors, especially on the higher levels of
the site and in the parent gateway, and \textit{(3)} to provide an image of a building,
office, device, or lab being discussed, especially in the School of Engineering
pages. These images are always useful and support the text around them.

\subsubsection*{Images --- Accessibility}

All images that would reasonably have captions do on MIT's website, background images do not but
adding captions to them would make the site less usable so that is a perfectly reasonable
decision. Only one of the images I looked at with any level of detail had a tooltip, and
about half of the images had alt text. When the alt text was present it was generally quite
useful, but it was not present several times when it would have been incredibly helpful to
someone who was blind, had a slow internet connection, or was dealing with tight filtering.

\subsubsection*{Presentation of Graphics}

The MIT site is incredibly minimalistic. It borders on dull while maintaining just enough flair
to seem interesting. This is obviously only considering the home portal. On average the
subordinate pages were much more interesting graphically with interactive menus,
three-dimensional backgrounds, and so on. There were some pages, the philosophy department's
page in particular, which were entirely drab as they were nothing but plaintext HTML with no
styling.

On the whole the site was somewhat dull, with interesting parts where it was likely many
people would be interacting, and incredibly plain parts where they were less likely to
frequent. This was highly dissappointing as MIT is considered one of the premiere
institutions of higher learning for technology and the sciences, but they cannot build an
attractive website.

\subsubsection*{Color}

Color was very minimal, most of the pages on the MIT site are white with black text. There
were colors on some pages, and when they were present they were tastefully presented. But
even with all the variance amoung the departments present on the site there were next to none
which employed color beyond the very minimum needed to differentiate menus from the main
body of their pages.

\subsubsection*{Audio and Video}

Sound was not present on the site, and beyond a few video clips from interviews, or the
showcasing of experiments on some research pages there was no video either. The videos
present were perfectly acceptable, and more often than not interesting, but there were very
few of them.

\subsubsection*{Advertising}

Advertisement was thankfully absent from the site. Given that the site is an infromation hub
for a very well-off private institution ads were wholly unnecessary.

\subsubsection*{Privacy and Monitoring}

The MIT site had no privacy policy per se, at least on the portions of the site accessible to the
general public, and thus this does not apply. Data collection was either nonexistant, or so
minimal as to escape detection even with repeated visits to the site.
