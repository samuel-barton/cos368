\section*{Input Features}

For both Stanford and MIT the best solution for testing their respective input handling capabilities was to execute
various searches utilizing each sites' local search engine. These searches ran the gamut from simple phrases to
Javascript code snippets.

\subsubsection*{Mechanism Undstandability and Usability}

            The searching functionality on the MIT site is very effective as the search
            bar is always located on the top right and is reasonably prominent relative
            to the rest of the page. Beyond that, the search space is determined by both
            one's current position in the site hierarchy, and, in some cases, the mode
            of search being done. From the home page there is the option to search the
            entire site, the `people' of MIT, or the offices of MIT. These detailed
            customizations of the search space make searching much more focused and useful.

            All of the input mechanisms for all four modes of the Stanford search
            engine are easy to use. The \textsl{Campus Map} context has significantly
            more widgets than the others, but because it is powered by Google Maps,
            it is designed very well and the functionality is familiar. All
            check boxes, radio buttons, and dropdown menus in all search contexts
            are clearly labeled.

\subsection*{Non-texual Input Supplementation}

            Radio buttons are used, as mentioned in the previous section, in a very useful
            way on the MIT site. The scope of the search is automatically chosen dependent
            on one's location in the sites hierarchy, but in those cases where further search
            customization would be helpful, radio buttons appear above the search bar to select
            search categories.

            In the Stanford search environment, all boolean modifiers appear as check
            boxes, and you can select between the \textsl{Web} and \textsl{People}
            contexts using radio buttons on the main page.

\subsection*{Validation}

            The MIT search engine will accept any kind of information, and in most cases will just
            search through the site for whatever is typed in; however, if one attempts to execute an
            attack via entering Javascript code into the search box (e.g. \texttt{javascript::alert(``Hello World!'')}),
            the site will not let the search go through but when the search is submitted will send
            the user to \url{http://web.mit.edu/cgi-bin/search-route-new} and tell them that they
            cannot access the page. There is no explanation anywhere as to what the user is expected to
            enter for data, and thus the user is left to their own devices to determine what they can or
            cannot search. Searching for non-alphabet characters \{!,@,[,*,/,\%,...\} has no adverse effect on
            the search experience, the search engine simply attempts to find relavent information containing
            those ``words''.

            While trying invalid input in the Stanford search engine, we discovered
            a peculiar error message. When entering a string of punctuation into
            the \textsl{People} or \textsl{Organizations} search field, the
            error message ``your search term must include more than 1 character''
            is returned. After some testing, it becomes clear that this is the case
            because the search function removes all punctuation characters before
            searching. This error message is a small oversight with no real
            negative side effects. However, we believe that it is bad form
            to have an error message that does not accurately reflect the input
            it is responding to.

