\section*{Summary and Conclusion}

Both the Stanford and MIT sites lacked any discernable indication of one's position within their respective hierarchies. 
There was no way to know what the parent page to the current page was, let alone trace the path down the hierarchy back to 
the home page. This made it so it was impossible to travel back to the current page's parent without using the browser's back 
button. This issue, while seemingly trivial, made navigation much more difficult as it was incredibly easy to dive down the 
hierarchy while looking for somethinkg, but then the only way to step back up to a previous level in the hierarchy was by returning
to the home page and starting the descent again. 

One issue where MIT suffered greately relative to Stanford was in site consistency. The MIT site was entirely inconsistent. The
home page was styled in one way, the page(s) for each department were styled in another way, and the individual sites for people 
were styled differently. Beyond this, there was a clear distinction in the quality of the design of the webpages between portions
of the site which the designers thought were likely to be visited often and those parts which were not. The portions of the site 
where people were likely to visit were incredibly complex and reasonably well thought out (see \url{https://sfs.mit.edu/}), 
but they stood in such contrast to the utterly minimal effort (see \url{http://web.mit.edu/philosophy/}) put in on designing other portions of the sight that they made the overall navigation experience worse. In order to fix their consistency 
issues, the designers of the MIT site should establish guidelines for colorschemes, layouts, and make a consistent header and 
footer to go on every page. This would not be that difficult to implement assuming a reasonable site design system. If on the other
hand, the site was designed in an entirely decentralized way, they could at least use a consistent logo and require similar colors.

The juxtaposition of the different portions of the site which happend while investigating the site emphasized just how bad parts 
of the site are by occasionally reminding the user how well designed pages look. Some of the pages (e.g. 
\url{http://www.eecs.mit.edu/}) on the EECS (Electrical Engineering and Computer Science) department's portion of the 
site were so poorly designed as to be unreadable. 

Stanford suffers from a complete lack of accessibility features for the images on the site with the exception of the home page. 