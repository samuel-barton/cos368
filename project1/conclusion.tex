\section*{Conclusion}

We unanimously agree that the Stanford website provides a more enjoyable experience
than the MIT site. Though they both lack navigational feedback, the
Stanford site is much easier to navigate due to its strict hierarchical structure
and consistency. Further, we felt that the consistency in both the Stanford brand
and look and feel was the strongest characteristic of the site. Despite being able
to accomplish our goal of learning about each university's
Master of Science program, the experience was much more enjoyable on Stanford's
site due to its consistency and style.

It is also easier to explore information on Stanford's site without having to
resort to the search engine. This is because of Stanford's effective use of
whitespace and hierarchical formatting. MIT's pages are organized like a
news website, with small, isolated blurbs covering the entire page. This makes
it easy to discover new information and interesting articles, but very
difficult to find specific information.

And this leads us to our final point. It is possible that the designers of the
MIT website chose the layout that they did because they had made different
assumptions about the user than we expected. It could be the case that they
wanted to make access to new and interesting articles related to MIT easy
and spontaneous, leaving information retrieval to the local search engine, or
to external search engines like Google. From this, we have learned a valuable lesson
about user interface design: sometimes the user's experience is outside of
your control. The motivations and goals of one user might differ drastically
from those of another, and these motivations will affect the user's personal
experience. From this, we believe that it is more important to address objective and
logical design decisions than to attempt to design the perfect user interface.