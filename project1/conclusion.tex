\section*{Summary and Conclusion}

There are some aspects of user interface design that both the Stanford and MIT
websites do particularly well. For example, both of the sites have exceptional
search engines. We found that we could find anything on the sites by using
these search engines, and we found the familiar Google output pleasing to
read. Neither of the search engines are perfect, but these imperfections were
mostly subjective stylistic choices.

The Stanford site excels at consistency. Until you get near the very bottom of the
site hierarchy, you can tell that you are on a Stanford page just by the use of
color and page layout alone. The Stanford University brand is reinforced by the
use of Stanford's colors in text and controls, as well as the tasteful images
of the university's campus. Interestingly, \textbf{MIT and Stanford have the
same university colors}. You wouldn't know this by visiting the MIT site, as
every page uses a different color scheme, and even the home page changes its
scheme every day.

During our time spent on the MIT website, we learned that \textit{community
matters to MIT}. This was obvious by the liberal use of headshots of people in
the MIT community on many of the site's pages. With this in mind, MIT handles
this emphasis exceptionally well. Images of people are very high quality and
have text captions and often \texttt{alt} attributes for each. Even the faculty
pages (for example, the faculty and advisors for the computer science department:
\url{https://www.eecs.mit.edu/people/faculty-advisors}) have images of most if
not all of the names on the page.

Stanford also includes images of faculty, staff, and students, but many less
than MIT. For example, Stanford's computer science faculty page
(\url{http://cs.stanford.edu/directory/faculty}) is a long list of names
with no images whatsoever. Even where there are images on Stanford's site,
there is a severe lack of accessibility features for them. Apart from those
on the home page, images on the Stanford site do not have tooltips, text
captions, or \texttt{alt} attributes.