\section*{Introduction}

In this paper we present the results of our analysis of two user interfaces.
We compared the user interfaces and user experience of the websites of two
reputable universities: Stanford University and the Massachusetts Institute
of Technology (MIT).

University websites are information hubs for a variety of visitors, each with
vast goals, interests, and motivations. It is critical that a university's
website provides a clear and navigable terrain, and that it provides effective
services for helping users find exactly the information they are looking for.
For example, a prospective student might visit a university website to learn about
offerings and opportunities, an enrolled student or alumnus might visit the site
to discover current events and university news, a parent might visit the site
to learn about tuition and fees, or campus life, safety, and services.

Designers and developers of university websites would be justified in making
the assumption that their users know what information they are looking for,
and that they are concerned only with finding that information. If a user
does not find what they were looking for, the website does not fulfill its
purpose.

We compared the Stanford and MIT websites against a set of criteria over
three sections: \textit{(1)} organization and navigation, \textit{(2)}
content and presentation of content, and \textit{(3)} input features.
To limit the scope of our analysis, we visited these websites as prospective
graduate students wanting to learn about Master of Science programs in
Computer Science.