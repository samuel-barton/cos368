\section*{Summary}

There are some aspects of user interface design that both the Stanford and MIT
websites do particularly well. For example, both of the sites have exceptional
search engines. We found that we could find anything on the sites by using
these search engines, and we found the familiar Google output pleasing to
read. Neither of the search engines are perfect, but these imperfections were
mostly subjective stylistic choices.

The Stanford site excels at consistency. Until you get near the very bottom of the
site hierarchy, you can tell that you are on a Stanford page just by the use of
color and page layout alone. The Stanford University brand is reinforced by the
use of Stanford's colors in text and controls, as well as the tasteful images
of the university's campus. Interestingly, \textbf{MIT and Stanford have the
same university colors}. You wouldn't know this by visiting the MIT site, as
every page uses a different color scheme, and even the home page changes its
scheme every day.

One issue where MIT suffers greatly relative to Stanford is in site consistency. The MIT site is entirely inconsistent. The
home page is styled in one way, the pages for each department are styled in another way, and the individual sites for people
are styled differently. Beyond this, there is a clear distinction in the quality of the design of the webpages between portions
of the site which the designers thought were likely to be visited often and those parts which were not. The portions of the site
where people are likely to visit are incredibly complex and reasonably well thought out (see \url{https://sfs.mit.edu/}),
but they stand in such contrast to the utterly minimal effort (see \url{http://web.mit.edu/philosophy/}) put in on designing other portions of the site that they made the overall navigation experience worse. In order to fix their consistency
issues, the designers of the MIT site should establish guidelines for color schemes and layouts, and make a consistent header and
footer to go on every page.

The juxtaposition of the different portions of the site emphasize just how bad parts
of the site are by occasionally reminding the user how well-designed pages look. Some of the pages (e.g.
\url{http://www.eecs.mit.edu/}) on the EECS (Electrical Engineering and Computer Science) department's portion of the
site are so poorly designed as to be unreadable.

During our time spent on the MIT website, we learned that \textit{community
matters to MIT}. This is obvious by the liberal use of headshots of people in
the MIT community on many of the site's pages. With this in mind, MIT handles
this emphasis exceptionally well. Images of people are very high quality and
have text captions and often \texttt{alt} attributes for each. Even the faculty
pages (for example, the faculty and advisors for the computer science department:
\url{https://www.eecs.mit.edu/people/faculty-advisors}) have images of most if
not all of the names on the page.

Stanford also includes images of faculty, staff, and students, but many less
than MIT. For example, Stanford's computer science faculty page
(\url{http://cs.stanford.edu/directory/faculty}) is a long list of names
with no images whatsoever. Even where there are images on Stanford's site,
there is a severe lack of accessibility features for them. Apart from those
on the home page, images on the Stanford site do not have tooltips, text
captions, or \texttt{alt} attributes.

Both the Stanford and MIT sites lack any discernable indication of one's position within their respective hierarchies.
There is no way to know what the parent page to the current page was, let alone trace the path down the hierarchy back to
the home page. This makes it impossible to travel back to the current page's parent without using the browser's back
button.