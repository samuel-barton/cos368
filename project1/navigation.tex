\section*{Organization and Navigation}
\subsubsection*{Organization Scheme}

Initially the MIT system is straightforward but it very quickly breaks down as while
the organization is hierarchical once you descend down into some sub portion of
the site, there is often no good way to get back to the home page, or even some
intermediary parent page. Some of the links are clearly labeled, but others are
nothing more than a hyperlinked number.

The home page is a portal from where you can get to wherever you need to go, and
from there the site becomes inconsistent. This is jarring at times, and it makes
navigating from one sublevel of the hierarchy to another without going back to the
parent level nearly, if not entirely, impossible.

On the other hand, the Stanford site provides a clean and effective hierarchical
organization. The system is split into three layers:
\begin{enumerate}
    \item In the upper-right corner the main page (\url{https://www.stanford.edu})
    there are four buttons: \textsl{STUDENTS}, \textsl{FACULTY/STAFF},
    \textsl{PARENTS}, and \textsl{ALUMNI}. These buttons lead to \textit{gateway}
    pages, which filter content from the site in the interest of the selected group.
    For example, if you are a student visiting the site, you can select the
    \textsl{STUDENTS} gateway to visit a page containing links to student-related content.

    The \textsl{STUDENTS} and \textsl{FACULTY/STAFF} gateway pages are extensions
    of the main site, and filter sections from \textsl{organization layer 2}
    verbatim. The \textsl{PARENTS} and \textsl{ALUMNI} gateway pages are actually
    independent sites: \url{https://parents.stanford.edu} and
    \url{https://alumni.stanford.edu}. The content on these sites are different
    than that on the main site, which greatly reduces clutter on the main page.

    \item The next level of organization is given by the tabs under the
    Stanford University logo on the first page. Under each of these tabs are
    sections containing links to other parts of the site. These sections are
    the same ones that are filtered by the \textsl{STUDENTS} and
    \textsl{FACULTY/STAFF} gateways.

    As an example, the ``About Stanford'' tab contains the section
    ``Stanford Facts \& History'', which itself contains a summary of the
    section and some links.

    \item The last level of organization is given by the links under the
    sections in \textsl{organization layer 2}. These links lead you deeper
    into the site.

    For almost every link, there is a page \texttt{$x$.stanford.edu} where
    $x$ is a brief descriptor of the page. For example, clicking the
    ``Stanford Facts'' link under the ``Stanford Facts \& History'' section
    from earlier will take you to \url{https://facts.stanford.edu}. This is
    where the hierarchical organization of the Stanford website begins to shine,
    as the site \url{https://facts.stanford.edu} is itself organized using these
    same three organization layers. This kind of consistency makes it really easy
    to find your way around most of the main page.
\end{enumerate}

\subsubsection*{Navigational Complexity}

On the MIT site, navigation down the hierarchy of pages is reasonably straightforward.
Information was easy to find, but once the information has been found navigating back
to where you came from is rather difficult. Even though the site is generally clear on
where to go to find information, searching for something specific would, in almost all
cases, yield the desired result much more quickly. A search will give the desired input
in one step, whereas navigation will take several steps in nearly all cases.

Information is easier to find by following links on the Stanford site, although
the search engine is top-notch as well. However, like the MIT site, there is some
information that is buried too far in the hierarchy to find without a search.
For example, tuition information is hard to find from the main page of the Stanford
site, but its page is the first hit after a search for ``tuition''.

\subsubsection*{Operation Feedback}

There is very little to no operation feedback present on both the MIT and Stanford sites.
There is no clear way to see where you have come from, or your current
position within the site's hierarchy beyond looking at the URL in your web browser. Thus
clicking on a link, while the animation of a hyperlink is still perfectly valid, does not
give any persistent information as to where you've come from once the new page has loaded.

\subsubsection*{Search Capabilities}

The MIT site has a very well thought out search engine which is local to the user's current
position within the site's hierarchy. Thus at the home page of the site the search engine
works across the entire MIT website, and at every department's main page the search is
across the department's pages. The responses of the search engine are both clear and
reasonably ordered.

Similarly, the Stanford site's search engine (which is powered by Google Custom
Search---go figure), is very impressive. Searches are fast, and the output format is
familiar to any Google user. On the main page, the search control in the top right
of the page has two modes selected by radio buttons: \textsl{Web} and \textsl{People}.
The search page \url{https://www.stanford.edu/search/} provides these modes as well
as two additional modes: \textsl{Organizations} and \textsl{Campus Map}.

The \textsl{Web} and \textsl{Campus Map} search environments are exceptionally good;
they are responsive and provide familiar output. However, the \textsl{People} and
\textsl{Organizations} search environments are quite poor. The tools are
cluttered on the page and there is too much information being displayed at once.
For example, a \textsl{People} search for ``Joe Schmo'', which has no results,
demonstrates how cluttered the interface is. At a glance, it is hard to tell
whether there were any results for the search or not because the page is too
messy, and the results string ``No Matches in Public Directory'' is too small.

Another downside of Stanford's \textsl{People} search environment is that a search
for something that's not a person does not invoke a suggestion to try a different
search environment. When searching for ``Computer Science'' under the \textsl{People}
environment, the engine does not suggest the possibility that the wrong search
environment was used, even though it provides other suggestions like trying
different filters or directories within the \textsl{People} environment.

\subsubsection*{Where Am I, and Can I Get Back to Where I Was?}

Unfortunately, the designers of both the MIT and Stanford websites neglected to
implement any form of position tracking, so getting lost is nearly effortless.
The only solution for getting back to earlier pages, which is generally implemented
but not always, is to go back to the root by locating, and clicking on, the MIT or
Stanford logo, which then takes you back to the home page where you can navigate
back to wherever you wish to go. This oversight makes the websites more
difficult to navigate, and is one of the downsides to their design.

\subsubsection*{Can I Get Home?}

On the MIT site, in most cases the MIT logo appears somewhere in the page's header, and clicking on that
will bring you back to the homepage. On those pages where the logo does not exist, as well as
on all other pages, there is a small footer at the very bottom of the page with the MIT logo
which will also bring you back to the home page.

Unlike the MIT site, Stanford does not always provide a way back home. The Stanford
logo appears in the header of some pages, which will bring you back to the home
page. Sometimes there is a link in the footer of a page that will bring you back
home as well. There are a handful of pages that do not link back to the main
site at all. These are usually pages for small organizations within a single
school in the university. This is probably because these pages are designed and
managed by students or professors who are not expected to adhere to any site
guidelines.

For both the MIT and Stanford sites, there should be at least a standard navigation
widget that must be included in every page as part of the guidelines. This would
provide consistency across the entire university site, even for smaller pages.

\subsubsection*{Appearance Consistency}

The main page of the MIT site is styled like a portal with the information displayed in such a way as to
only use up one screen's worth of space. Each child page is formatted slightly differently.
Departments each have their own unique theme which the majority of their pages hold to, and
individual faculty often have a personalized page with another different style.

The constant changes in page styling, combined with the lack of a consistent header and
menu, were jarring. As one navigates through the site it would be easy to initially
come to the incorrect conclusion that they have been redirected away from MIT's site entirely.

The Stanford site has a more consistent appearance overall. Individual schools and
departments differ a bit from the main page, but the use of university colors and imagery
reinforces the university's brand throughout. Pages for smaller organizations do
vary completely from the university's look and feel, but it is not jarring---these
pages are usually well-designed for their own purposes. An example of one of these
smaller pages is for the Stanford InfoLab: \url{https://infolab.stanford.edu}.

\subsubsection*{Human Error in Site Development}

Both the Stanford and MIT websites are massively complex and thus some human error is unsurprising.
That being said, MIT has significantly less broken links than Stanford. We used the
World Wide Web Consortium's Link Checker tool (\url{https://validator.w3.org/checklink})
to test both of the websites. Here is a table of the results:

\begin{center}
\begin{tabular}{|l|c|c|}
\hline
                                     & \textbf{MIT} & \textbf{Stanford} \\ \hline
\textbf{404} Not Found               & 1            & 3                 \\ \hline
\textbf{403} Forbidden               & 2            & 11                \\ \hline
\textbf{405} Method Not Allowed      & 0            & 1                 \\ \hline
\textbf{500} Server Error            & 0            & 1                 \\ \hline \hline
\multicolumn{1}{|r|}{\textbf{Total}} & 3            & 16                \\ \hline
\end{tabular}
\end{center}

\pagebreak
\subsubsection*{Purpose Realization}

The task of our visit to these sites was to learn about the opportunities for
graduate study in computer science at MIT and Stanford.

For MIT, completing this task was difficult. Learning about the opportunities for
graduate and doctoral study ranged in difficulty from determining the various areas
of study within computer science one could specialize in for a graduate or doctoral
degree, which was reasonably challenging but doable with a nontrivial amount of effort,
to determining how much it would cost to get either a master's or doctoral degree,
which was not possible within the several hours spent navigating through the site
assuming one limited oneself to only using the MIT website, and its searching
functionalities, and not emailing faculty or staff.

The difficulty of finding this information, which would be useful to many individuals, was
quite dissappointing. This information should have been accessible with minimal effort, and
despite the overal good design of the site's hierarchy it was not.

The designers of the MIT site, specifically the site designed for the Computer Science Department,
should have required a more readable and accessible envronment to display information. A better
thought out menu would have been enormously helpful, and having clearly differentiated pages with
relavent information on them would also have helped. More focus on the information needed by new
prospective students rather than the individual members of the department, be they students or
faculty, would have made the process of learning about MIT and what it has to offer much more
straightforward.

For Stanford, completing our task was a delight. All information regarding
graduate computer science studies are available from the Stanford School of
Engineering hub. The computer science department page clearly lists
all specializations for graduate study, with each section containing example
courses, involved faculty contact information, and even external resources. This
information is found at
\url{https://www-cs.stanford.edu/academics/current-masters/choosing-specialization}.

