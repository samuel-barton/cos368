\section*{Organization and Navigation}
\subsection*{Site Organization --- Clear or Intractable?}

Initially the MIT system is straightforward but it very quickly breaks down as while
the organization is hierarchical once you descend down into some sub portion of
the site, there is often no good way to get back to the home page, or even some
intermediary parent page. Some of the links are clearly labeled, but others are
nothing more than a hyperlinked number.

The home page is a portal from where you can get to wherever you need to go, and
from there the site becomes inconsistent. This is jarring at times, and it makes
navigating from one sublevel of the hierarchy to another without going back to the
parent level nearly, if not entirely, impossible.

On the other hand, the Stanford site provides a clean and effective hierarchical
organization. The system is split into three layers:
\begin{enumerate}
    \item In the upper-right corner the main page (\url{https://www.stanford.edu})
    there are four buttons: \textsl{STUDENTS}, \textsl{FACULTY/STAFF},
    \textsl{PARENTS}, and \textsl{ALUMNI}. These buttons lead to \textit{gateway}
    pages, which filter content from the site in the interest of the selected group.
    For example, if you are a student visiting the site, you can select the
    \textsl{STUDENTS} gateway to visit a page containing links to student-related content.

    The \textsl{STUDENTS} and \textsl{FACULTY/STAFF} gateway pages are extensions
    of the main site, and filter sections from \textsl{organization layer 2}
    verbatim. The \textsl{PARENTS} and \textsl{ALUMNI} gateway pages are actually
    independent sites: \url{https://parents.stanford.edu} and
    \url{https://alumni.stanford.edu}. The content on these sites are different
    than that on the main site, which greatly reduces clutter on the main page.

    \item The next level of organization is given by the tabs under the
    Stanford University logo on the first page. Under each of these tabs are
    sections containing links to other parts of the site. These sections are
    the same ones that are filtered by the \textsl{STUDENTS} and
    \textsl{FACULTY/STAFF} gateways.

    As an example, the ``About Stanford'' tab contains the section
    ``Stanford Facts \& History'', which itself contains a summary of the
    section and some links.

    \item The last level of organization is given by the links under the
    sections in \textsl{organization layer 2}. These links lead you deeper
    into the site.

    For almost every link, there is a page \texttt{$x$.stanford.edu} where
    $x$ is a brief descriptor of the page. For example, clicking the
    ``Stanford Facts'' link under the ``Stanford Facts \& History'' section
    from earlier will take you to \url{https://facts.stanford.edu}. This is
    where the hierarchical organization of the Stanford website begins to shine,
    as the site \url{https://facts.stanford.edu} is itself organized using these
    same three organization layers. This kind of consistency makes it really easy
    to find your way around most of the main page.
\end{enumerate}